\documentclass{article}

\usepackage{amsmath, amsfonts, amsthm} 
\usepackage{bm}
\usepackage{listings}
\usepackage{graphicx}
\usepackage{float}
\usepackage{subfigure}
\usepackage{geometry}

\geometry{
	paper=a4paper, 
	top=2.5cm,
	bottom=2.5cm, 
	left=2.5cm, 
	right=3cm,
	headsep=0.75cm, 
}
\title{Cheetah MPC}
\author{Zhuang Yulun}
\date{\today}

\begin{document}

\maketitle

\section{Conventions}


\begin{tabular}{l l}
    \hline
    Notation & \\
    \hline
    $ i$ & ith leg\\
    $_\mathcal B$ &left subscript for body coordinate system\\
    $[\mathbf x]_\times \in \mathbb R^{3\times 3}$ & skew-symmetric matrix, $[\mathbf x]_\times \mathbf y = \mathbf x \times \mathbf y$\\
    $\mathbf \Theta = [\phi, \theta, \psi]^T$ & orientation as ZYX Euler angles\\
    $\mathbf J_i\in \mathbb R^{3\times 3}$ & foot Jacobain\\
    $\mathbf\Lambda_i\in \mathbb R^{3\times 3}$ &operational space inertia matrix\\
    $\mathbf K_p, \mathbf K_d\in \mathbb R^{3\times 3}$ &diagonal positive definite pd gain matrices\\
    \hline

\end{tabular}

\section{Dynamics}

\begin{align}
    \ddot{\mathbf{p}} &=\frac{\sum_{i=1}^{n} \mathbf{f}_{i}}{m}-\mathbf{g} \\
    \frac{\mathrm{d}}{\mathrm{d} t}(\mathbf{I} \boldsymbol{\omega}) &=\sum_{i=1}^{n} \mathbf{r}_{i} \times \mathbf{f}_{i} \\
    \dot{\mathbf{R}} &=[\boldsymbol{\omega}]_{\times} \mathbf{R}
\end{align}

\begin{align}
    \mathbf{R} &=\mathbf{R}_{z}(\psi) \mathbf{R}_{y}(\theta) \mathbf{R}_{x}(\phi) \\
    \frac{\mathrm{d}}{\mathrm{d} t}(\mathbf{I} \boldsymbol{\omega}) &=\mathbf{I} \dot{\boldsymbol{\omega}}+\boldsymbol{\omega} \times(\mathbf{I} \boldsymbol{\omega}) \approx \mathbf{I} \dot{\boldsymbol{\omega}}
\end{align}

$$
\boldsymbol{\omega}=\left[\begin{array}{ccc}
    \cos (\theta) \cos (\psi) & -\sin (\psi) & 0 \\
    \cos (\theta) \sin (\psi) & \cos (\psi) & 0 \\
    -\sin (\theta) & 0 & 1
    \end{array}\right]\left[\begin{array}{l}
    \dot{\phi} \\
    \dot{\theta} \\
    \dot{\psi}
\end{array}\right]
$$

\begin{align}
    \left[\begin{array}{l}
    \dot{\phi} \\
    \dot{\theta} \\
    \dot{\psi}
    \end{array}\right] &=\left[\begin{array}{ccc}
    \cos (\psi) / \cos (\theta) & \sin (\psi) / \cos (\theta) & 0 \\
    -\sin (\psi) & \cos (\psi) & 0 \\
    \cos (\psi) \tan (\theta) & \sin (\psi) \tan (\theta) & 1
    \end{array}\right] \boldsymbol{\omega} \\
    & \approx\left[\begin{array}{ccc}
    \cos (\psi) & \sin (\psi) & 0 \\
    -\sin (\psi) & \cos (\psi) & 0 \\
    0 & 0 & 1
    \end{array}\right] \boldsymbol{\omega} \approx \mathbf{R}_{z}^{\top}(\psi) \boldsymbol{\omega}
\end{align}

\section{State Space Model}

\begin{align}
    &\frac{\mathrm d}{\mathrm d t} \hat{\mathbf\Theta}=R_{z}(\psi) \hat{\mathbf\Theta} \\
    &\frac{\mathrm d}{\mathrm d t} \hat{\mathbf p}=\hat{\dot{\mathbf p}} \\
    &\frac{\mathrm d}{\mathrm d t} \hat{\mathbf \omega}=\hat{\mathbf I}^{-1} \sum_{i=1}^{n} \mathbf r_{i} \times \mathbf f_{i}=\hat{\mathbf I}^{-1}\left(\left[\mathbf r_{1}\right]_{\times} \mathbf f_{1}+\cdots+\left[\mathbf r_{n}\right]_{\times} \mathbf f_{n}\right) \\
    &\frac{\mathrm d}{\mathrm d t} \hat{\dot{\mathbf p}}=\frac{\sum_{i=1}^{n} \mathbf f_{i}}{m}-\mathbf g=\frac{\mathbf f_{1}+\cdots+\mathbf f_{n}}{m}-\mathbf g
\end{align}


\begin{align}
    \frac{\mathrm{d}}{\mathrm{d} t}\left[
    \begin{array}{c}
        \hat{\mathbf\Theta} \\
        \hat{\mathbf{p}} \\
        \hat{\mathbf{\omega}} \\
        \hat{\dot{\mathbf{p}}}\\
        \mathbf g
    \end{array}
    \right]=
    \left[
    \begin{array}{lllll}
        \mathbf{0}_{3} & \mathbf{0}_{3} & \mathbf{R}_{z}(\psi) & \mathbf{0}_{3} & \mathbf 0_{3\times 1}\\
        \mathbf{0}_{3} & \mathbf{0}_{3} & \mathbf{0}_{3} & \mathbf{I}_{3} & \mathbf 0_{3\times 1}\\
        \mathbf{0}_{3} & \mathbf{0}_{3} & \mathbf{0}_{3} & \mathbf{0}_{3} & \mathbf 0_{3\times 1}\\
        \mathbf{0}_{3} & \mathbf{0}_{3} & \mathbf{0}_{3} & \mathbf{0}_{3} & [0\ 0\ 1]^T\\
        \mathbf{0}_{1\times 3} & \mathbf{0}_{1\times 3} & \mathbf{0}_{1\times 3} & \mathbf{0}_{1\times 3} & 0
    \end{array}
    \right]&\left[
    \begin{array}{c}
        \hat{\mathbf{\Theta}} \\
        \hat{\mathbf{p}} \\
        \hat{\mathbf \omega} \\
        \hat{\dot{\mathbf{p}}}\\
        \mathbf g
    \end{array}\right]\\
    +\left[
    \begin{array}{lll}
        \mathbf{0}_{3} & \cdots & \mathbf{0}_{3} \\
        \mathbf{0}_{3} & \cdots & \mathbf{0}_{3} \\
        \hat{\mathbf{I}}^{-1}\left[\mathbf{r}_{1}\right]_{\times} & \cdots & \hat{\mathbf{I}}^{-1}\left[\mathbf{r}_{4}\right]_\times \\
        \mathbf{I}_{3} / m & \cdots & \mathbf{I}_{3} / m\\
        \mathbf{0}_{1\times 3}&\cdots&\mathbf{0}_{1\times 3}
    \end{array}
    \right]
    &\left[\begin{array}{c}
    \mathbf{f}_{1} \\
    \vdots \\
    \mathbf{f}_{4}
    \end{array}\right]
\end{align}

$$
\dot{\mathbf{x}}(t)=\mathbf{A}_{c}(\psi) \mathbf{x}(t)+\mathbf{B}_{c}\left(\mathbf{r}_{1}, \ldots, \mathbf{r}_{n}, \psi\right) \mathbf{u}(t)
$$
where $\mathbf{A}_{c} \in \mathbb{R}^{13 \times 13}$ and $\mathbf{B}_{c} \in \mathbb{R}^{13 \times 3 n}$


\section{ZOH Discretization}

Given a linear system $\dot{x} = Ax+Bu$ and its solution 
$$
\mathbf{x}(t)=e^{\mathbf{A}\left(t-t_{0}\right)} \mathbf{x}\left(t_{0}\right)+\int_{t_{0}}^{t} e^{\mathbf{A}(t-\tau)} \mathbf{B u}(\tau) d \tau
$$
Let $t_{0}=t_{k}, t=t_{k+1}$, and $t_{k+1}-t_{k}=T$. Therefore,
$$
\begin{aligned}
\mathbf{x}_{k+1} &=e^{\mathbf{A} T} \mathbf{x}_{k}+\int_{t_{k}}^{t_{k+1}} e^{\mathbf{A}\left(t_{k+1}-\tau\right)} \mathbf{B u}(\tau) d \tau \\
&=e^{\mathbf{A} T} \mathbf{x}_{k}+\int_{t_{k}}^{t_{k+1}} e^{\mathbf{A}\left(t_{k+1}-\tau\right)} \mathbf{B} d \tau \mathbf{u}_{k}
\end{aligned}
$$
Let $\tau^{\prime}=t_{k+1}-\tau$, we have
$$
\int_{t_{k}}^{t_{k+1}} e^{\mathbf{A}\left(t_{k+1}-\tau\right)} \mathbf{B} d \tau=\int_{T}^{0} e^{\mathbf{A} \tau^{\prime}}\left(-d \tau^{\prime}\right) \mathbf{B}=\int_{0}^{T} e^{\mathbf{A} \tau} d \tau \mathbf{B}
$$
Thus,
$\mathbf{x}_{k+1} =\mathbf{A}_{d} \mathbf{x}_{k}+\mathbf{B}_{d} \mathbf{u}_{k}$
, where 
$\mathbf{A}_{d} =e^{\mathbf{A} T}$ 
and 
$\mathbf{B}_{d} =\int_{0}^{T} e^{\mathbf{A} \tau} d \tau \mathbf{B}$

\begin{align}
    \mathbf{x}_{k+1} &=\mathbf{A}_{d} \mathbf{x}_{k}+\mathbf{B}_{d} \mathbf{u}_{k} \\
    \mathbf{A}_{d} &=e^{\mathbf{A} T} \\
    \mathbf{B}_{d} &=\int_{0}^{T} e^{\mathbf{A} \tau} d \tau \mathbf{B}
\end{align}

\section{MPC Formulation}

$$
\min _{\mathbf{x}, \mathbf{u}} \sum_{i=0}^{k-1}\left\|\mathbf{x}_{i+1}-\mathbf{x}_{i+1, \mathrm{ref}}\right\|_{\mathbf{Q}_{i}}+\left\|\mathbf{u}_{i}\right\|_{\mathbf{R}_{i}}
$$

$$
subject \  to \quad \mathbf{x}_{i+1}=\mathbf{A}_{i} \mathbf{x}_{i}+\mathbf{B}_{i} \mathbf{u}_{i}, i=0 \ldots k-1
$$
$$
\underline{\mathbf{c}}_{i} \leq \mathbf{C}_{i} \mathbf{u}_{i} \leq \overline{\mathbf{c}}_{i}, i=0 \ldots k-1
$$
$$
\mathbf{D}_{i} \mathbf{u}_{i}=0, i=0 \ldots k-1
$$

$$
\begin{array}{r}
    f_{\min } \leq f_{z} \leq f_{\max } \\
    -\mu f_{z} \leq \pm f_{x} \leq \mu f_{z} \\
    -\mu f_{z} \leq \pm f_{y} \leq \mu f_{z}
\end{array}
$$

\section{QP Formulation}
$$
\begin{aligned}
    \bm{x_k} &= \mathbf A^k\bm x_0 + \sum_{i=0}^{k-1}\mathbf{A}^{k-1-i}\mathbf B \mathbf u_i\\
    \left[
        \begin{array}{c}
            \bm x_1\\
            \bm x_2\\
            \vdots \\
            \bm x_k
        \end{array}
    \right]
    &=
    \left[
        \begin{array}{c}
            \mathbf A^1\\
            \mathbf A^2\\
            \vdots\\
            \mathbf A^k
        \end{array}
    \right]
    \bm x_0
    +
    \left[
        \begin{array}{ccccc}
            \bm 0 & & &\cdots & \bm 0\\
            \mathbf B & \bm 0 && \cdots & \bm 0\\
            \mathbf{AB} & \mathbf B & \bm 0 & \cdots & \bm 0\\
            \vdots & & & \ddots & \vdots\\
            \mathbf{A}^{k-1}\mathbf B & \mathbf{A}^{k-1}\mathbf B & \bm 0 & \cdots & \mathbf B
        \end{array}
    \right]
    \left[
        \begin{array}{c}
            \mathbf u_0\\
            \mathbf u_1\\
            \vdots\\
            \mathbf u_{k-1}\\
        \end{array}
    \right]
\end{aligned}
$$

$$
\begin{array}{lr}
    \min _{\mathbf{U}} & \frac{1}{2} \mathbf{U}^{\boldsymbol{\top}} \mathbf{H} \mathbf{U}+\mathbf{U}^{\boldsymbol{\top}} \mathbf{g} \\
    \text { s. t. } & \underline{\mathbf{c}} \leq \mathbf{C U} \leq \overline{\mathbf{c}}
\end{array}
$$

\begin{align}
    \mathbf{H} &=2\left(\mathbf{B}_{\mathrm{qp}}^{\boldsymbol{L}} \mathbf{\mathbf { B } _ { \mathrm { qp } }}+\mathbf{K}\right) \\
    \mathbf{g} &=2 \mathbf{B}_{\mathrm{qp}}^{\top} \mathbf{L}\left(\mathbf{A}_{\mathrm{qp}} \mathbf{x}_{0}-\mathbf{y}\right)
\end{align} 

\section{Ground Normal Estimation}

To enable the ability to traverse stairs and sloped terrain without vision, we use
measurements of each footstep location $$\boldsymbol{p}_i=(p^{x}_i, p^{y}_i, p^{z}_i)$$
to approximate the local slope of the walking surface.
\\
The walking surface is modeled as a plane:
$$
z(x, y)=a_{0}+a_{1} x+a_{2} y
$$
\\
Coefficients $\boldsymbol{a}=\left(a_{0}, a_{1}, a_{2}\right)^{T}$ are obtained through the solution of the least squares problem
$$
\begin{aligned}
\boldsymbol{a} &=\left(\boldsymbol{W}^{T} \boldsymbol{W}\right)^{\dagger} \boldsymbol{W}^{T} \boldsymbol{p}^{z} \\
\boldsymbol{W} &=\left[\begin{array}{lll}
\mathbf{1} & \boldsymbol{p}^{x} & \boldsymbol{p}^{y}
\end{array}\right]_{4 \times 3}
\end{aligned}
$$
\\
where $\boldsymbol{p}^{x}=(p^{x}_1, p^{x}_2, p^{x}_3, p^{x}_4)$

\end{document}
